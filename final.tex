\documentclass[12pt]{article}
\usepackage{amsmath}
\usepackage{amssymb}

\newcommand{\Z}{\mathbb{Z}}
\newcommand{\R}{\mathbb{R}}
\newcommand{\C}{\mathbb{C}}
\newcommand{\del}{\partial}

\begin{document}

\title{Topology II: Differential Topology\\Final Examination} 
\date{December 8, 2008}
\maketitle


You may use the statement of an earlier problem even if you do not prove it. Each question is worth $5$ points unless explicitly stated otherwise. The maximum score is $55$.

\section*{Part A}

Let $\psi:S^n\to S^n$ be a smooth map from the $n$-dimensional sphere $S^n\subset \R^{n+1}$ to itself. We say that $\psi$ is \emph{even} if $\psi(-x)=\psi(x)$ for all $x\in S^n$. We say that $\psi$ is \emph{odd} if $\psi(-x)=-\psi(x)$ for all $x\in S^n$.

\begin{enumerate}

\item Show that the (oriented) degree of $\psi$ is even if and only if for a regular value $y$ for $\psi$, the cardinality of $\psi^{-1}(y)$ is even.
\item Suppose $\psi$ is an even map. Show that the degree of $\psi$ is an even integer.
\item\label{mu} If $\psi:S^n\to S^n$ is a map such that for all $x\in S^n$, $\psi(-x)\neq -\psi(x)$, (i.e., no pair of antipodal points map to antipodal points), then show that we have a well-defined \emph{even} map $\mu:S^n\to S^n$ given by
$$\mu(x)=\frac{\psi(x)+\psi(-x)}{\Vert\psi(x)+\psi(-x) \Vert}$$
\item For $\psi$ as in Problem~\ref{mu}, show that $\psi$ is homotopic to $\mu$.
\item Deduce that if $\psi:S^n\to S^n$ has odd degree, then for some $x\in S^n$, we have $\psi(-x)=-\psi(x)$.

\end{enumerate}

\section*{Part B}

Let $T$ be the vector space of linear functions $\del:C^\infty(\R^n)\to\R$ such that 
$$\del(fg)=f(0)\del(g)+\del(f)g(0).$$

\begin{enumerate}
\item Show that for $1\leq i\leq n$, the map $\del_i:f\mapsto \frac{\del f(0)}{\del x_i}$ is in $T$ and for the coordinate functions $x_j$, 
$$\del_i x_j=\delta_{ij}$$
where $\delta_{ij}$ is the Kronecker delta.
\item Deduce that $\del_i$ are linearly independent elements of $T$.
\item Show that every $\del\in T$ is of the form 
$$\del=\sum_{i=1}^n a_i\del_i$$
for some constants $a_i\in\R$. Deduce that the elements $\del_i$, $1\leq i\leq n$, form a basis for $T$.
\item Consider the ideal $m$ in $C^\infty(\R^n)$ given by
$$m=\{f\in C^\infty(\R^n):f(0)=0\}.$$
Show that if $\del\in T$ and $\del f=0\ \forall f\in m$, then $\del=0$ (i.e., $\del g=0$ for all $g\in C^\infty(\R^n)$). 

\item Let $T^*$ denote the dual of $T$. Show that the map 
$$\mu:m\to T^*$$
given by associating to $f\in m$ the linear functional $\mu(f)\in T^*$
such that 
$$\mu(f):\del\mapsto \del(f)$$
is surjective.  

\item ($10=2+8$ points) Show that the kernel of the map 
$$\mu:m\to T^*$$ 
is the ideal $m^2$ in $C^\infty(\R^n)$ generated by the set
$$\{fg:f,g \in m\}$$

 Thus $T^*$ is isomorphic to $m/m^2$, the \emph{Zariski cotangent space}.

\end{enumerate}



\end{document}
